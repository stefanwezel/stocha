%Template
% !TeX spellcheck = de 
\documentclass[a4paper]{scrartcl}
\usepackage[utf8]{inputenc}
%\usepackage[ngerman]{babel}
\usepackage{geometry,forloop,fancyhdr,fancybox,lastpage}
\usepackage{dsfont}

\usepackage{listings}
\lstset{frame=tb,
	language=Java,
	aboveskip=3mm,
	belowskip=3mm,
	showstringspaces=false,
	columns=flexible,
	basicstyle={\small\ttfamily},
	numbers=left,
	numberstyle=\tiny\color{gray},
	keywordstyle=\color{blue},
	commentstyle=\color{dkgreen},
	stringstyle=\color{mauve},
	breaklines=true,
	breakatwhitespace=true,
	tabsize=3
}
\geometry{a4paper,left=3cm, right=3cm, top=3cm, bottom=3cm}
% Diese Daten müssen pro Blatt angepasst werden:
\newcommand{\NUMBER}{6}
\newcommand{\EXERCISES}{3}
% Diese Daten müssen einmalig pro Vorlesung angepasst werden:
\newcommand{\COURSE}{Stochastik}
\newcommand{\TUTOR}{TBD}
\newcommand{\STUDENTA}{Stefan Wezel}
\newcommand{\STUDENTB}{Stefan Wezel}
\newcommand{\STUDENTC}{}
\newcommand{\DEADLINE}{07.06.2018}




%Math
\usepackage{amsmath,amssymb,tabularx}

%Figures
\usepackage{graphicx,tikz,color,float}
\graphicspath{ {home/stefan/picures/} }
\usetikzlibrary{shapes,trees}

%Algorithms
\usepackage[ruled,linesnumbered]{algorithm2e}

%Kopf- und Fußzeile
\pagestyle {fancy}
\fancyhead[L]{\STUDENTA}
\fancyhead[C]{\COURSE}
\fancyhead[R]{\today}

\fancyfoot[L]{}
\fancyfoot[C]{}
\fancyfoot[R]{Seite \thepage}

%Formatierung der Überschrift, hier nichts ändern
\def\header#1#2{
	\begin{center}
		{\Large\bf Übungsblatt #1}\\
		{(Abgabetermin #2)}
	\end{center}
}

%Definition der Punktetabelle, hier nichts ändern
\newcounter{punktelistectr}
\newcounter{punkte}
\newcommand{\punkteliste}[2]{%
	\setcounter{punkte}{#2}%
	\addtocounter{punkte}{-#1}%
	\stepcounter{punkte}%<-- also punkte = m-n+1 = Anzahl Spalten[1]
	\begin{center}%
		\begin{tabularx}{\linewidth}[]{@{}*{\thepunkte}{>{\centering\arraybackslash} X|}@{}>{\centering\arraybackslash}X}
			\forloop{punktelistectr}{#1}{\value{punktelistectr} < #2 } %
			{%
				\thepunktelistectr &
			}
			#2 &  $\Sigma$ \\
			\hline
			\forloop{punktelistectr}{#1}{\value{punktelistectr} < #2 } %
			{%
				&	
			} &\\
			\forloop{punktelistectr}{#1}{\value{punktelistectr} < #2 } %
			{%
				&
			} &\\
		\end{tabularx}
	\end{center}
}

\begin{document}

\section*{Aufgabe 9}
Wir können folgenden Wahrscheinlichkeitsraum benutzen:\\\\
$
\Omega = \lbrace 
	\text{rot ohne Aufdruck}, 
	\text{rot mit Aufdruck},\\
	\text{grün ohne Aufdruck},
	\text{grün mit Aufdruck},\\
	\text{blau ohne Aufdruck},
	\text{blau mit Aufdruck}
	\rbrace
\\\\
\mathcal{F} = \mathcal{P}(\Omega)\\\\
P = \lbrace
	P(\emptyset) = 0,\\
	P(\text{rot ohne Aufdruck}) = \frac{27}{100},\\
	P(\text{rot mit Aufdruck}) = \frac{3}{100},\\
	P(\text{grün ohne Aufdruck}) = \frac{12}{50},\\
	P(\text{grün mit Aufdruck}) = \frac{3}{50},\\
	P(\text{blau ohne Aufdruck}) = \frac{4}{100},\\
	P(\text{blau mit Aufdruck}) = \frac{36}{100},\\
	P(\Omega) = 1
	\rbrace
\\\\
... = (\Omega, \mathcal{F}, P)
$
\\\\
Nach dem Theorem von Bayes können wir für die Wahrscheinlichkeit, dass eine gezogene Kugel rot ist, unter Annahme des Priors, dass sie ein Aufdruck besitzt folgende Gleichung aufstellen:\\\\
$
P(Rot|Aufdruck) \Leftrightarrow \frac{P(Aufdruck|Rot) \cdot P(Rot)}{P(Aufdruck)}\\\\
\Leftrightarrow \frac{P(Aufdruck|Rot) \cdot P(Rot)}{P(Aufdruck|Rot) \cdot P(Rot) + P(Aufdruck|Gruen) \cdot P(Gruen) + P(Aufdruck|Blau) \cdot P(Blau)}\\\\
\Leftrightarrow \frac{\frac{1}{10} \cdot \frac{3}{10}}{\frac{1}{10} \cdot \frac{3}{10} + \frac{1}{5} \cdot \frac{3}{10} + \frac{1}{10} \cdot \frac{4}{10}}\\\\
\Leftrightarrow \frac{\frac{3}{100}}{\frac{3}{100} + \frac{6}{100} + \frac{4}{100}}\\\\
\Leftrightarrow \frac{\frac{3}{100}}{\frac{13}{100}}\\\\
\Leftrightarrow \frac{300}{1300}\\\\
\Leftrightarrow \frac{3}{13}
$


\end{document}
