%Template
% !TeX spellcheck = de 
\documentclass[a4paper]{scrartcl}
\usepackage[utf8]{inputenc}
%\usepackage[ngerman]{babel}
\usepackage{geometry,forloop,fancyhdr,fancybox,lastpage}
\usepackage{dsfont}
\usepackage{tikz}
\usepackage[utf8]{inputenc}
%\usepackage[T1]{fontec}
\usepackage{lmodern}
\usepackage{helvet}
\usepackage{geometry}
\usepackage{mathptmx}
\usepackage{amsmath}
\usepackage{amssymb}
\usepackage{graphicx}
\usepackage{tabularx}
\usepackage{ragged2e}
\usepackage{array}
\usepackage[ngerman]{babel}
%\usepackage[table,dvipsnames,svgnames]{xcolor}
\usepackage{lscape}

\usepackage{listings}
\lstset{frame=tb,
	language=Java,
	aboveskip=3mm,
	belowskip=3mm,
	showstringspaces=false,
	columns=flexible,
	basicstyle={\small\ttfamily},
	numbers=left,
	numberstyle=\tiny\color{gray},
	keywordstyle=\color{blue},
	commentstyle=\color{dkgreen},
	stringstyle=\color{mauve},
	breaklines=true,
	breakatwhitespace=true,
	tabsize=3
}
\geometry{a4paper,left=3cm, right=3cm, top=3cm, bottom=3cm}
% Diese Daten müssen pro Blatt angepasst werden:
\newcommand{\NUMBER}{6}
\newcommand{\EXERCISES}{3}
% Diese Daten müssen einmalig pro Vorlesung angepasst werden:
\newcommand{\COURSE}{Stochastik}
\newcommand{\TUTOR}{TBD}
\newcommand{\STUDENTA}{Stefan Wezel}
\newcommand{\STUDENTB}{Stefan Wezel}
\newcommand{\STUDENTC}{}
\newcommand{\DEADLINE}{07.06.2018}




%Math
\usepackage{amsmath,amssymb,tabularx}

%Figures
\usepackage{graphicx,tikz,color,float}
\graphicspath{ {home/stefan/picures/} }
\usetikzlibrary{shapes,trees}

%Algorithms
\usepackage[ruled,linesnumbered]{algorithm2e}

%Kopf- und Fußzeile
\pagestyle {fancy}
\fancyhead[L]{\STUDENTA}
\fancyhead[C]{\COURSE}
\fancyhead[R]{\today}

\fancyfoot[L]{}
\fancyfoot[C]{}
\fancyfoot[R]{Seite \thepage}

%Formatierung der Überschrift, hier nichts ändern
\def\header#1#2{
	\begin{center}
		{\Large\bf Übungsblatt #1}\\
		{(Abgabetermin #2)}
	\end{center}
}

%Definition der Punktetabelle, hier nichts ändern
\newcounter{punktelistectr}
\newcounter{punkte}
\newcommand{\punkteliste}[2]{%
	\setcounter{punkte}{#2}%
	\addtocounter{punkte}{-#1}%
	\stepcounter{punkte}%<-- also punkte = m-n+1 = Anzahl Spalten[1]
	\begin{center}%
		\begin{tabularx}{\linewidth}[]{@{}*{\thepunkte}{>{\centering\arraybackslash} X|}@{}>{\centering\arraybackslash}X}
			\forloop{punktelistectr}{#1}{\value{punktelistectr} < #2 } %
			{%
				\thepunktelistectr &
			}
			#2 &  $\Sigma$ \\
			\hline
			\forloop{punktelistectr}{#1}{\value{punktelistectr} < #2 } %
			{%
				&	
			} &\\
			\forloop{punktelistectr}{#1}{\value{punktelistectr} < #2 } %
			{%
				&
			} &\\
		\end{tabularx}
	\end{center}
}

\begin{document}

\section*{Aufgabe 12}
Um die Teilmengen festzulegen, wollen wir zunaechst herausfinden, was die Umkehrfunktion unserer Zufallsvariable ist. Setzen wir dann die Intervallgrenzen in die Umkehrfunktion ein, so bekommen wir die gesuchten Teilmengen von $\Omega$.
\subsection*{(a)}
Für das Intervall $\left[X < 0 \right]$:\\
\text{Negative Werte muessen hier nicht beruecksichtigt werden, da wir uns im reellen Zahlenraum befinden.}\\
\\
Für das Intervall $\left[ 1 \leq x \leq 2 \right]$ setzen wir jetzt die Grenzen 1 und zwei ein. Die Ergebnisse bilden dann die Grenzen des Intervall in der Grundemenge.\\
$
X(\omega) = 1 \Rightarrow \omega = 1^2 \Rightarrow \pm 1 \Rightarrow \lbrace x:-1 \leq x \leq 1 \rbrace \subseteq \Omega\\
X(\omega) = 2 \Rightarrow \omega = 2^2 \Rightarrow \pm 4 \Rightarrow \lbrace x: -4 \leq x \leq 4 \rbrace \subseteq \Omega
$\\
\\
Die Teilmenge ist nun $\lbrace x: -4 \leq x \leq 4 \rbrace $, da sie die andere enthält.

\subsection*{(b)}
\begin{align*}
	F_x(t) &= P(\left[x \leq t \right])\\
	&=P(\lbrace \omega \in \Omega: \sqrt{|\omega|} \leq t)\\
	\text{1. Fall:}&\\
	t < 0:&\\
	&\rightarrow \lbrace \omega \in \Omega: \sqrt{|\omega|} < t\rbrace\\
	&\rightarrow F_x(t) = P(\emptyset) = 0\\
	\\
	\text{2. Fall:}&\\
	t \geq 0&\\
	&\rightarrow \lbrace \omega \in \Omega: \sqrt{|\omega|} \leq t\rbrace &= \lbrace \omega \in \Omega: -t^2 \leq \omega \leq t^2\rbrace\\
	&= \left[-t^2, t^2 \right]\\
	&\rightarrow F_x(t) = P(\left[-t^2, t^2 \right])\\
	&=P(\left[-t^2, t^2\right] \cap \left[ -4, 4 \right])\\
	& = P(\left[-2, 2 \right])
\end{align*}













\end{document}
