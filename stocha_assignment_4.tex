%Template
% !TeX spellcheck = de 
\documentclass[a4paper]{scrartcl}
\usepackage[utf8]{inputenc}
%\usepackage[ngerman]{babel}
\usepackage{geometry,forloop,fancyhdr,fancybox,lastpage}
\usepackage{dsfont}
\usepackage{tikz}
\usepackage[utf8]{inputenc}
%\usepackage[T1]{fontec}
\usepackage{lmodern}
\usepackage{helvet}
\usepackage{geometry}
\usepackage{mathptmx}
\usepackage{amsmath}
\usepackage{amssymb}
\usepackage{graphicx}
\usepackage{tabularx}
\usepackage{ragged2e}
\usepackage{array}
\usepackage[ngerman]{babel}
%\usepackage[table,dvipsnames,svgnames]{xcolor}
\usepackage{lscape}

\usepackage{listings}
\lstset{frame=tb,
	language=Java,
	aboveskip=3mm,
	belowskip=3mm,
	showstringspaces=false,
	columns=flexible,
	basicstyle={\small\ttfamily},
	numbers=left,
	numberstyle=\tiny\color{gray},
	keywordstyle=\color{blue},
	commentstyle=\color{dkgreen},
	stringstyle=\color{mauve},
	breaklines=true,
	breakatwhitespace=true,
	tabsize=3
}
\geometry{a4paper,left=3cm, right=3cm, top=3cm, bottom=3cm}
% Diese Daten müssen pro Blatt angepasst werden:
\newcommand{\NUMBER}{6}
\newcommand{\EXERCISES}{3}
% Diese Daten müssen einmalig pro Vorlesung angepasst werden:
\newcommand{\COURSE}{Stochastik}
\newcommand{\TUTOR}{TBD}
\newcommand{\STUDENTA}{Stefan Wezel}
\newcommand{\STUDENTB}{Stefan Wezel}
\newcommand{\STUDENTC}{}
\newcommand{\DEADLINE}{07.06.2018}




%Math
\usepackage{amsmath,amssymb,tabularx}

%Figures
\usepackage{graphicx,tikz,color,float}
\graphicspath{ {home/stefan/picures/} }
\usetikzlibrary{shapes,trees}

%Algorithms
\usepackage[ruled,linesnumbered]{algorithm2e}

%Kopf- und Fußzeile
\pagestyle {fancy}
\fancyhead[L]{\STUDENTA}
\fancyhead[C]{\COURSE}
\fancyhead[R]{\today}

\fancyfoot[L]{}
\fancyfoot[C]{}
\fancyfoot[R]{Seite \thepage}

%Formatierung der Überschrift, hier nichts ändern
\def\header#1#2{
	\begin{center}
		{\Large\bf Übungsblatt #1}\\
		{(Abgabetermin #2)}
	\end{center}
}

%Definition der Punktetabelle, hier nichts ändern
\newcounter{punktelistectr}
\newcounter{punkte}
\newcommand{\punkteliste}[2]{%
	\setcounter{punkte}{#2}%
	\addtocounter{punkte}{-#1}%
	\stepcounter{punkte}%<-- also punkte = m-n+1 = Anzahl Spalten[1]
	\begin{center}%
		\begin{tabularx}{\linewidth}[]{@{}*{\thepunkte}{>{\centering\arraybackslash} X|}@{}>{\centering\arraybackslash}X}
			\forloop{punktelistectr}{#1}{\value{punktelistectr} < #2 } %
			{%
				\thepunktelistectr &
			}
			#2 &  $\Sigma$ \\
			\hline
			\forloop{punktelistectr}{#1}{\value{punktelistectr} < #2 } %
			{%
				&	
			} &\\
			\forloop{punktelistectr}{#1}{\value{punktelistectr} < #2 } %
			{%
				&
			} &\\
		\end{tabularx}
	\end{center}
}

\begin{document}

\section*{Aufgabe 12}
\subsection*{(a)}
Um die Teilmengen festzulegen, wollen wir zunaechst herausfinden, was die Umkehrfunktion unserer Zufallsvariable ist. Setzen wir dann die Intervallgrenzen in die Umkehrfunktion ein, so bekommen wir die gesuchten Teilmengen von $\Omega$.

Für das Intervall $\left[X < 0 \right]$:\\
\text{Negative Werte muessen hier nicht beruecksichtigt werden, da wir uns im reellen Zahlenraum befinden.}\\
\\
Für das Intervall $\left[ 1 \leq x \leq 2 \right]$ setzen wir jetzt die Grenzen 1 und 2 ein. Die Ergebnisse bilden dann die Grenzen des Intervall in der Grundemenge.\\\\
$
X(\omega) = 1 \Rightarrow \omega = 1^2 \Rightarrow \pm 1 \Rightarrow \lbrace x:-1 \leq x \leq 1 \rbrace \subseteq \Omega\\
X(\omega) = 2 \Rightarrow \omega = 2^2 \Rightarrow \pm 4 \Rightarrow \lbrace x: -4 \leq x \leq 4 \rbrace \subseteq \Omega
$
\subsection*{(b)}
Für die Verteilungsfunktion unterscheiden wir folgende 3 Fälle.\\
\\
$
F_x(t) = P(\left[x \leq t \right])\\
=P(\lbrace \omega \in \Omega: \sqrt{|\omega|} \leq t)\\
\\
F(t) = 
\begin{cases}
\text{1. Fall, falls } t < -4:\\
\text{2. Fall, falls } -4 \leq t < 4\\
\text{3.Fall, falls } 4 < t
\end{cases}
$\\\\
Nun berechnen wir die einzelnen Fälle:\\
\begin{align*}
	\text{1. Fall:}&\\
	t < -4:&\\
	F(t) &= P(\left(-\infty, t\right]) = \int_{-\infty}^{t} f(x)dx = 0\\
	\text{2. Fall:}&\\
	-4 \leq t < 4&\\
	F(t) &= P(\left(-\infty, t\right]) = \int_{-\infty}^{t} f(x)dx + \int_{-4}^{t}f(x)dx = 0 + \sqrt{|t|}\\
	\text{3.Fall:}&\\
	4 \leq t&\\
	F(t) &= P(\left(-\infty, t\right]) =\int_{-\infty}^{t} f(x)dx + \int_{-4}^{4}f(x)dx + \int_{4}^{t}f(x)dx\\
	&=0+1+0 = 1
\end{align*}\\
\\
Die Dichtefunktion $f_X(x)$ ergibt sich aus der Ableitung der gegebenen Verteilungsfunktion. Um die Intervallgrenzen miteinzubeziehen nutzen wir die Indikatorfunktion.\\\\
$
F'(x) = f(x) = \frac{d}{dx}\sqrt{|x|} = \frac{1}{2}x^{-\frac{1}{2}} \cdot \frac{d}{dx}|x| = \frac{1}{2}x^{-\frac{1}{2}} \cdot \frac{x}{|x|}
$




\section*{Aufgabe 13}
\subsection*{(a)}
Um die c zu bestimmen berechnen wir mithilfe von r zunächst das Integral der Funktion $f$ mit den Grenzen 0 und 4. Zunächst noch ohne c. Wir wissen, dass die Fläche unter einer Dichtefunktion 1 sein muss. Haben wir also das Integral berechnet normieren wir das Ergebnis, indem wir 1 durch es teilen. Daraus ergibt sich der Wert $0.5015351$ für c.\\
Dafür kann folgender R code verwendet werden:\\
\\
\begin{lstlisting}
	eq = function(x){x^(-x)}
	area = integrate(eq, 0, 4)
	c = 1/as.numeric(area3[1])
	print(c)
	>>> [1] 0.5015351 # <- Ergebnis fuer c
\end{lstlisting}



\subsection*{(b)}
Um die Verteilungsfunktion für $f(x) = 0.5015351 \cdot x^{-x} \mathds{1}_{\left[0,4\right]}(x)$ aufzustellen, unterscheiden wir drei Fälle. \\
Für $t < 0$:\\
$
F(t) = P(\left(-\infty, t  \right]) = \int_{-\infty}^{t}f(x)dx = \int_{-\infty}^{t} = 0\\
$\\
Für $ 0 \leq t < 4$:\\
$
F(t) = P(\left(-\infty, t  \right]) = \int_{-\infty}^{0}f(x)dx + \int_{0}^{t} f(x)dx= \int_{-\infty}^{t}  + \int_{0}^{t}0.5015351 \cdot x^{-x}= 0 + \int_{0}^{t}0.5015351 \cdot\\
$\\
Für $ 4 < t$:\\
$
F(t) = P(\left(-\infty, t  \right]) = \int_{-\infty}^{0}f(x)dx + \int_{0}^{4} f(x)dx + \int_{4}^{t}f(x)dx= \int_{-\infty}^{4}  + \int_{0}^{4}\;0.5015351 \cdot x^{-x} + \int_{4}^{t}= 0 + 1 + 0 = 1\\
$\\
\\
In R sieht die Verteilungsfunktion, dann folgendermaßen aus:\\
\\
\begin{lstlisting}
F <- function(t){
	if(t < 0){
		return(0)
	}
	else if(t >=0 && t < 4){
		return(as.numeric(integrate(density, lower = 0, upper = t)[1]))
	}
	else {
		return(1)
	}
}
\end{lstlisting}

\begin{figure}[H]
	\includegraphics*[scale = 1]{blatt_4_2b.png}
	\caption{Die Verteilungsfunktion für $f(x) = 0.5015351 \cdot x^{-x} \mathds{1}_{\left[0,4\right]}(x)$.}
\end{figure}



\subsection*{(c)}
\begin{figure}[H]
	\includegraphics*[scale = 0.5]{blatt_4_13_c.png}
	\caption{Die Inverse der Verteilungsfunktion für $f(x) = 0.5015351 \cdot x^{-x} \mathds{1}_{\left[0,4\right]}(x)$.
	(Mit ggplot hat es leider nicht geklappt...)}
\end{figure}

Die Inverse und der Plot wurden mit folgendem Code erstellt:\\
\begin{lstlisting}
	Finv <- Vectorize(function (s)uniroot((function (x) F(x) - s), lower = 0, upper = 4)[1])
	curve(Finv, 0,1)
\end{lstlisting}

\newpage

\section*{Aufgabe 14}
\subsection*{(b)}
Mithilfe der Gleichung $\omega = X_0(\omega) \cdot 1 + X_1(\omega)\cdot 2$ und dem Wissen, dass $X_0, X_1$ entweder 0 oder 1 sind, können wir kleine Gleichungssysteme aufstellen, und so die Verteilung herausfinden.\\

\begin{align*}
	0 &= X_0(0) \cdot 1 + X_1(0) \cdot 2\\
	&= 0 \cdot 1 + 0 \cdot 2\\
	\\
	1 &= X_0(1) \cdot 1 + X_1(1) \cdot 2\\
	&= 1 \cdot 1 + 0 \cdot 2\\
	\\
	2 &= X_0(2) \cdot 1 + X_1(2) \cdot 2\\
	&= 0 \cdot 1 + 1 \cdot 2\\
	\\
	3 &= X_0(3) \cdot 1 + X_1(3) \cdot 2\\
	&= 1 \cdot 1 + 1 \cdot 2\\
	\\
\end{align*}\\
Die Verteilung können wir als Tabelle darstellen:\\\\
\begin{tabular}{|c|c|c|c|c|}
	\hline 
	$\omega$ & 0 & 1 & 2 & 3 \\ 
	\hline 
	$X_0(\omega)$ & 0 & 1 & 0 & 1 \\ 
	\hline 
	$X_1(\omega)$ & 0 & 0 & 1 & 1 \\ 
	\hline 
\end{tabular} \\\\\\
Daraus ergibt sich folgende Verteilung:\\
\\
\begin{align*}
	X_0(\omega) = 
	\begin{cases}
	\text{1, falls } \omega \in \lbrace 1,3 \rbrace\\
	\text{0, falls } \omega \in \lbrace 0,2 \rbrace
	\end{cases}\\\\
	X_1(\omega) = 
	\begin{cases}
	\text{1, falls } \omega \in \lbrace 2,3 \rbrace\\
	\text{0, falls } \omega \in \lbrace 0,1 \rbrace
	\end{cases}
\end{align*}
\newpage
\subsection*{(b)}
Für $X_0 \in B_0$ und $X_1 \in B_1$ stellen wir nun Abbildungen $A_0$ und $A_1$ auf.\\
Für $B_0 = B_1 = \lbrace 0 \rbrace$\\
$A^0_{B_0} = \lbrace 0, 2 \rbrace$\\
und: \\
$A^1_{B_1} = \lbrace 0,1 \rbrace$\\
$A^0_{B_0} \cap A^1_{B_1} = \lbrace 0 \rbrace$\\
$\Rightarrow P(\lbrace 0 \rbrace) = P(\lbrace 0,2 \rbrace)P(\lbrace 0,1 \rbrace)$
$\Rightarrow$ unabhängig $\checkmark$
\\
\\
\\
Für $B_0 = B_1 = \lbrace 1 \rbrace$\\
$A^0_{B_0} = \lbrace 1,3 \rbrace$\\
und: \\
$A^1_{B_1} = \lbrace 2,3 \rbrace$\\
$A^0_{B_0} \cap A^1_{B_1} = \lbrace 3 \rbrace$\\
$\Rightarrow P(\lbrace 3 \rbrace) = P(\lbrace 1,3 \rbrace)P(\lbrace 2,3 \rbrace)$
$\Rightarrow$ unabhängig $\checkmark$
\\
\\
\\
Für $B_0 = \lbrace 1 \rbrace , B_1 = \lbrace 0 \rbrace$\\
$A^0_{B_0} = \lbrace 1,3 \rbrace$\\
und: \\
$A^1_{B_1} = \lbrace 0,1 \rbrace$\\
$A^0_{B_0} \cap A^1_{B_1} = \lbrace 1 \rbrace$\\
$\Rightarrow P(\lbrace 1 \rbrace) = P(\lbrace 1,3 \rbrace)P(\lbrace 0,1 \rbrace)$
$\Rightarrow$ unabhängig $\checkmark$
\\
\\
\\
Für $B_0 = \lbrace 0 \rbrace , B_1 = \lbrace 1 \rbrace$\\
$A^0_{B_0} = \lbrace 0,2 \rbrace$\\
und: \\
$A^1_{B_1} = \lbrace 2,3 \rbrace$\\
$A^0_{B_0} \cap A^1_{B_1} = \lbrace 2 \rbrace$\\
$\Rightarrow P(\lbrace 2 \rbrace) = P(\lbrace 0,2 \rbrace)P(\lbrace 2,3 \rbrace)$
$\Rightarrow$ unabhängig $\checkmark$\\
\\
Da alle Ereignisse unabhängig sind, sind es auch die beiden Zufallsvariablen $X_0$ und $X_1$. $\square$



\end{document}
