%Template
% !TeX spellcheck = de 
\documentclass[a4paper]{scrartcl}
\usepackage[utf8]{inputenc}
%\usepackage[ngerman]{babel}
\usepackage{geometry,forloop,fancyhdr,fancybox,lastpage}
\usepackage{dsfont}

\usepackage{listings}
\lstset{frame=tb,
	language=Java,
	aboveskip=3mm,
	belowskip=3mm,
	showstringspaces=false,
	columns=flexible,
	basicstyle={\small\ttfamily},
	numbers=left,
	numberstyle=\tiny\color{gray},
	keywordstyle=\color{blue},
	commentstyle=\color{dkgreen},
	stringstyle=\color{mauve},
	breaklines=true,
	breakatwhitespace=true,
	tabsize=3
}
\geometry{a4paper,left=3cm, right=3cm, top=3cm, bottom=3cm}
% Diese Daten müssen pro Blatt angepasst werden:
\newcommand{\NUMBER}{6}
\newcommand{\EXERCISES}{3}
% Diese Daten müssen einmalig pro Vorlesung angepasst werden:
\newcommand{\COURSE}{Stochastik}
\newcommand{\TUTOR}{TBD}
\newcommand{\STUDENTA}{Stefan Wezel}
\newcommand{\STUDENTB}{Stefan Wezel}
\newcommand{\STUDENTC}{}
\newcommand{\DEADLINE}{07.06.2018}




%Math
\usepackage{amsmath,amssymb,tabularx}

%Figures
\usepackage{graphicx,tikz,color,float}
\graphicspath{ {home/stefan/picures/} }
\usetikzlibrary{shapes,trees}

%Algorithms
\usepackage[ruled,linesnumbered]{algorithm2e}

%Kopf- und Fußzeile
\pagestyle {fancy}
\fancyhead[L]{\STUDENTA}
\fancyhead[C]{\COURSE}
\fancyhead[R]{\today}

\fancyfoot[L]{}
\fancyfoot[C]{}
\fancyfoot[R]{Seite \thepage}

%Formatierung der Überschrift, hier nichts ändern
\def\header#1#2{
	\begin{center}
		{\Large\bf Übungsblatt #1}\\
		{(Abgabetermin #2)}
	\end{center}
}

%Definition der Punktetabelle, hier nichts ändern
\newcounter{punktelistectr}
\newcounter{punkte}
\newcommand{\punkteliste}[2]{%
	\setcounter{punkte}{#2}%
	\addtocounter{punkte}{-#1}%
	\stepcounter{punkte}%<-- also punkte = m-n+1 = Anzahl Spalten[1]
	\begin{center}%
		\begin{tabularx}{\linewidth}[]{@{}*{\thepunkte}{>{\centering\arraybackslash} X|}@{}>{\centering\arraybackslash}X}
			\forloop{punktelistectr}{#1}{\value{punktelistectr} < #2 } %
			{%
				\thepunktelistectr &
			}
			#2 &  $\Sigma$ \\
			\hline
			\forloop{punktelistectr}{#1}{\value{punktelistectr} < #2 } %
			{%
				&	
			} &\\
			\forloop{punktelistectr}{#1}{\value{punktelistectr} < #2 } %
			{%
				&
			} &\\
		\end{tabularx}
	\end{center}
}

\begin{document}

\section*{Aufgabe 7}
\subsection*{(a)}
$
\;\;\;\;\;\int_{-1}^{1} c(-x^4 + 1) \cdot \mathds{1}_{\left[-1,2\right]}\\\\
\Leftrightarrow \int_{-1}^{1} c(-x^4 + 1)\\\\
\Leftrightarrow \left[ c(-\frac{1}{5} \cdot x^5 + x) \right]_{-1}^1\\\\
\Leftrightarrow \left( c \left( -\frac{1}{5} + 1 \right)\right) - \left( c \left( \frac{1}{5} - 1 \right)  \right)\\\\
\Leftrightarrow c \left( \frac{4}{5} \right) - c\left( -\frac{4}{5} \right)\\\\
\Leftrightarrow c \left( \frac{4}{5} + \frac{4}{5} \right)\\\\
\Leftrightarrow \frac{8}{5}  \cdot c
$\\\\
Wir wissen: das Integral muss 1 ergeben.\\
$
\Rightarrow \frac{8}{5} \cdot c = 1\\
\Leftrightarrow c = \frac{5}{8} 
$\\
Ein geeignets $c$ währe also $c=\frac{5}{8}$\\
\\
Auf das selbe Ergebnis kommt man, wenn folgenden Code ausführt:\\
Hierbei ignorieren wir zunächst die Konstante c um sie dann anschließend gleich der bekannten Fläche 1 zu setzen.
\begin{lstlisting}
	eq = function(x){(1 - x^4)}
	
	area = (integrate(eq, -1, 1))
	c = (1 / area) # normalize area under curve
	print(c)
	>>> [1] 0.625 # =5/8
\end{lstlisting}


\subsection*{(b)}
%TODO
Die Verteilungsfunktion $F$ ist die Stammfuntion von $f$. Daher gilt:\\
$F = \frac{5}{8} \cdot (\frac{1}{5}x^5 + x)$
\subsection*{(c)}
\begin{center}
	\includegraphics*[scale = 0.5]{aufgabe_7_c.png}
\end{center}


\subsection*{(c)}
$
\;\;\;\;\;\int_{-0.5}^{0.5} \frac{5}{8}(-x^4 + 1) \cdot \mathds{1}_{\left[-1,2\right]}\\\\
\Leftrightarrow \int_{-0.5}^{0.5} \frac{5}{8}(-x^4 + 1)\\\\
\Leftrightarrow \left[ \frac{5}{8} (\frac{1}{5} x^5 + x) \right]_{-\frac{1}{2}}^{\frac{1}{2}}\\\\
\Leftrightarrow 
\frac{5}{8}(-\frac{1}{5} \cdot \frac{1}{32} + \frac{1}{2})
- \frac{5}{8}(\frac{1}{5} \cdot -\frac{1}{32} - \frac{1}{2})\\\\
\Leftrightarrow 
\frac{5}{8}(-\frac{1}{160} + \frac{80}{160})
- \frac{5}{8}(\frac{1}{160} - \frac{80}{160})\\\\
\Leftrightarrow \frac{5}{8} \cdot \frac{79}{160} - \frac{5}{8} \cdot - \frac{79}{160}\\\\
\Leftrightarrow\frac{5}{8} \cdot \frac{158}{160}\\\\
\Leftrightarrow\frac{79}{128} = 0.6171875
$
\\
\\
Durch folgenden R code  lässt sich die Wahrscheinlichkeit bestimmen:\\
\begin{lstlisting}
	eq = function(x){0.625 * (1 - x^4)}
	probability = integrate(eq2, -0.5, 0.5)
	print(probability)
	>>> [1] 0.6171875
\end{lstlisting}







\section*{Aufgabe 8}
\subsection*{(a)}
\begin{center}
	\includegraphics*[scale = 0.5]{aufgabe_8_a.png}
\end{center}


\subsection*{(b)}
Der Korrelationskoeffizient nach Pearson von Größe und Masse ist $0.9307936$. Die Variablen Masse und Größe korrelieren also stark miteinander. Die lässt sich auch in der Abbildung unten erkennen. Die einzelnen roten Punkte (die jeweils für Samples stehen) sind nah an der blauen linearen Regressionslinie.
\begin{center}
	\includegraphics*[scale = 0.5]{aufgabe_8_b.png}
\end{center}


\subsection*{(c)}
$\Omega = \lbrace f,m\rbrace$\\
$\mathcal{F} = \lbrace \emptyset, \lbrace f \rbrace, \lbrace m\rbrace, \Omega\rbrace$\\
$ P(\emptyset) = \frac{0}{18} = 0, P(m) = \frac{10}{18} = \frac{5}{9}, P(f) = \frac{8}{18} = \frac{4}{9}, P(\Omega) = \frac{18}{18} = 1$


\subsection*{(d)}
$\Omega = \lbrace21,22,23,24,25,26,27,28 \rbrace$\\
$\mathcal{F} = \mathcal{P}(\Omega)$\\
$ P=unif(\Omega)$

%TODO check whether this is correct
\begin{center}
	\includegraphics*[scale = 0.5]{alter_verteilungsfunktion.png}
\end{center}


\subsection*{(e)}
Die Wahrscheinlichkeit, dass eine zufällig gewählte Person weiblich ist und braune Haare hat lässt sich aus dem gegebenen Datensatz ablesen. Darin gibt es lediglich zwei Personen die sowohl weiblich sind also auch braune Haare haben. Diese sind Emma und Julia. Die Wahrscheinlichkeit entspricht also der, dass entweder Emma oder Julia ausgewählt werden, also $\frac{1}{18} + \frac{1}{18} = \frac{2}{18} = \frac{1}{9}$. \\
Läge uns der Datensatz nicht vor, sondern nur die Information, dass 8 von 18 Personen weiblich sind und 6 von 18 Personen braune Haare haben könnten wir die Wahrscheinlichkeit nicht genau bestimmen.

\end{document}
